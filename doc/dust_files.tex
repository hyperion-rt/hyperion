\documentclass[11pt]{article}
\usepackage[left=2.0cm,top=3.0cm,right=2.0cm,bottom=3.0cm,nohead,a4paper]{geometry}
\renewcommand{\familydefault}{\sfdefault}

\setlength{\parindent}{0pt}

\RequirePackage[calcwidth]{titlesec}
\titleformat{\subsection}[block]{\sffamily\bfseries}{\large}{12pt}{\\\large}[{\titlerule[0.5pt]}]

\begin{document}

\title{Dust files for RT Code}
\author{Thomas Robitaille}
\date{2 December 2009}
\maketitle

This document describes the format required of `dust files' for the RT code. The format adopted for all files is the \texttt{FITS} standard.

\section{Dust files}

\subsection{HDU 0}

The primary header must contain the following keywords:

\begin{verbatim}
VERSION  = 1          integer
TYPE     = 1          integer
EMISSVAR = `T'        string
\end{verbatim}

Where the \texttt{TYPE} argument can take the following values:

\begin{verbatim}
TYPE     = 1:         Scattering elements are specified by a four-element matrix
\end{verbatim}

and the \texttt{EMISSVAR} argument can take the following values:

\begin{verbatim}
EMISSVAR = `T':       Emissivity is a function of temperature
EMISSVAR = `J':       Emissivity is a function of J_mean
\end{verbatim}

\subsection{HDU 1 - \texttt{extname=OPTICAL PROPERTIES}}

This HDU should contain a table the following columns:

\begin{verbatim}
nu               Frequency (Hz)
albedo           Albedo
kappa            Opacity (m^2/kg)
P1, P2, P3, P4   Scattering Matrix elements
\end{verbatim}

The scattering matrix elements should be given by vector columns, where each row contains the matrix elements as a function of scattering angle $\mu$.

\subsection{HDU 2 - \texttt{extname=SCATTERING ANGLES}}

This HDU contains a table with a single column, \texttt{mu}, which are the cosines of the scattering angles for which the matrix elements are tabulated.

\subsection{HDU 3 - \texttt{extname=MEAN OPACITIES}}

This HDU contains a table with three columns:

\begin{verbatim}
temperature      The blackbody temperature (K)
kappa_plank      Plank mean opacity (m^2/kg)
kappa_rosseland  Rosseland mean opacity (m^2/kg)
\end{verbatim}

\subsection{HDU 4 - \texttt{extname=EMISSIVITIES}}

This HDU contains two columns:

\begin{verbatim}
nu               Frequency (Hz)
jnu              Emissivity (?)
\end{verbatim}

The emissivity is a vector column, where each row contains the emissivity as a function of an independent variable (e.g. temperature or mean intensity)

\subsection{HDU 5 - \texttt{extname=EMISSIVITY VARIABLE}}

This table contains the independent variable for the emissivities. This can be \texttt{temperature} if \texttt{EMISSVAR='T'} or \texttt{jmean} if \texttt{EMISSVAR='J'}.

\section{Grid files}

\subsection{HDU 0}

The primary header must contain a 2 or 3D array containing e.g. the density, temperature, mass, volume, etc.

\subsection{HDU 1}

Contains a table with a single column giving the wall positions along the first dimension (should have size NAXIS1 + 1).

\subsection{HDU 2}

Contains a table with a single column giving the wall positions along the second dimension (should have size NAXIS2 + 1).

\subsection{HDU 3 (optional)}

Contains a table with a single column giving the wall positions along the first dimension (should have size NAXIS3 + 1).

\end{document}
